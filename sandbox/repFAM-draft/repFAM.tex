\documentclass[12pt]{article} % 12-point font

\usepackage[margin=1in]{geometry} % set page to 1-inch margins
\usepackage{amsmath} % for math
\usepackage{amssymb} % like \Rightarrow
\usepackage{bm} % bold math
\setlength\parindent{0pt} % Suppresses the indentation of new paragraphs.

% Bibliography
\usepackage{natbib}
\bibliographystyle{plainnat}

% Big display
\newcommand{\ds}{ \displaystyle }
% Parenthesis
\newcommand{\norm}[1]{\left\lVert#1\right\rVert}
\newcommand{\p}[1]{\left(#1\right)}
\newcommand{\bk}[1]{\left[#1\right]}
\newcommand{\bc}[1]{ \left\{#1\right\} }
\newcommand{\abs}[1]{ \left|#1\right| }
% Derivatives
\newcommand{\df}[2]{ \frac{d#1}{d#2} }
\newcommand{\ddf}[2]{ \frac{d^2#1}{d{#2}^2} }
\newcommand{\pd}[2]{ \frac{\partial#1}{\partial#2} }
\newcommand{\pdd}[2]{\frac{\partial^2#1}{\partial{#2}^2} }
% Distributions
\newcommand{\Normal}{ \text{Normal} }
\newcommand{\Beta}{ \text{Beta} }
\newcommand{\G}{ \text{Gamma} }
\newcommand{\InvGamma}{ \text{Inv-Gamma} }
\newcommand{\Uniform}{ \text{Uniform} }
% Statistics
\newcommand{\E}{ \text{E} }
\newcommand{\iid}{\overset{iid}{\sim}}
\newcommand{\ind}{\overset{ind}{\sim}}

% Adds settings for hyperlinks. (Mainly for table of contents.)
\usepackage{hyperref}
\hypersetup{
  pdfborder={0 0 0} % removes red box from links
}

% Title Settings
\title{Template Doc}
\author{Arthur Lui}
\date{18 October, 2018} % \date{} to set date to empty

% MAIN %
\begin{document}

\maketitle

\tableofcontents \newpage % Comment to remove table of contents

\section{A section}\label{sec:a-section} % add labels to refer to other sections easily
\section{Repulsive Feature Allocation Model (rep-FAM)}{sec:repfam}

% IDEAS: HMC?

Recall that a $J \times K$ binary matrix $Z$ characterizes $K$ different cell phenotypes.
Let $v_k \mid \alpha \iid \Beta(\alpha, 1)$, $k = 1, \dots , K$.
We define a joint distribution of $Z = \bk{z_1, \dots, z_K}$ as

$$
P(\bm Z \mid \bm v , C_\phi) \propto \bc{\prod_{k=1}^K \prod_{j=1}^J v_k^{z_{jk}} (1 - v_k)^{1 - z_{jk}}}
\times
\bc{\prod_{k_1=1}^{K-1} \prod_{k_2=k_1+1}^K 1 - C_\phi(\rho(\bm z_{k1}, \bm z_{k2}))}
$$

where $\rho(\bm z_{k_1}, \bm z_{k_2})$ measure distance between columns $k_1$ and
$k_2$, for $k_1 \neq k_2$, and $C_\phi(\cdot)$ is a continuous decreasing
function in distance with $C_\phi(0)=1$ and
$\lim_{d\rightarrow\infty}C_\phi(d)= 0$. For a distance metric, we use
$\rho(\bm z_{k_1}, \bm z_{k_2})=\sum_{j=1}^J \abs{z_{jk_1} - z_{jk_2}}$, the number
of discordances between columns $k_1$ and $k_2$. The function $C_\phi(\cdot)$
can be interpreted as a proximity function. A suitable form is $C_\phi(d) =
\exp\p{-d/\phi}$. \cite{quinlan2017parsimonious} showed that the model in
\eqref{eq:rep-FAM} has a finite normalizing constant and the distribution is
proper.

% Under the model in \eqref{eq:rep-FAM}, probability 0 is assigned to
% $\bm Z$ having identical columns.  For matrices $\bm Z$ that have the same number
% of 0's and 1's, $\bm Z$ with similar columns has a smaller probability since
% $C_\phi(\cdot)$ is decreasing in distance. $C_\phi(\cdot)$ smoothly penalizes
% any $\bm Z$ having similar columns in the prior and can remove redundant columns in
% posterior inference. Note that different from the IBP, under the model in
% \eqref{eq:rep-FAM}, $\Prob(z_{jk}=1) \neq v_k$ due to the repulsive function.
% We place a prior on $\alpha$, such that $\alpha \sim \text{Gamma}(a_\alpha,
% b_\alpha)$.

\bibliography{bib}
\end{document}
