\documentclass[12pt]{article} % 12-point font

\usepackage[margin=1in]{geometry} % set page to 1-inch margins
\usepackage{amsmath} % for math
\usepackage{amssymb} % like \Rightarrow
\setlength\parindent{0pt} % Suppresses the indentation of new paragraphs.

% Big display
\newcommand{\ds}{ \displaystyle }
% Parenthesis
\newcommand{\norm}[1]{\left\lVert#1\right\rVert}
\newcommand{\p}[1]{\left(#1\right)}
\newcommand{\bk}[1]{\left[#1\right]}
\newcommand{\bc}[1]{ \left\{#1\right\} }
\newcommand{\abs}[1]{ \left|#1\right| }
% Derivatives
\newcommand{\df}[2]{ \frac{d#1}{d#2} }
\newcommand{\ddf}[2]{ \frac{d^2#1}{d{#2}^2} }
\newcommand{\pd}[2]{ \frac{\partial#1}{\partial#2} }
\newcommand{\pdd}[2]{\frac{\partial^2#1}{\partial{#2}^2} }
% Distributions
\newcommand{\Normal}{ \text{Normal} }
\newcommand{\Beta}{ \text{Beta} }
\newcommand{\G}{ \text{Gamma} }
\newcommand{\InvGamma}{ \text{Inv-Gamma} }
\newcommand{\Uniform}{ \text{Uniform} }
% Statistics
\newcommand{\E}{ \text{E} }
\newcommand{\iid}{\overset{iid}{\sim}}
\newcommand{\ind}{\overset{ind}{\sim}}

% Graphics
\usepackage{graphicx}  % for figures
\usepackage{float} % Put figure exactly where I want [H]

% Uncomment if using bibliography
% Bibliography
% \usepackage{natbib}
% \bibliographystyle{plainnat}

% Adds settings for hyperlinks. (Mainly for table of contents.)
\usepackage{hyperref}
\hypersetup{
  pdfborder={0 0 0} % removes red box from links
}

% Title Settings
\title{TSNE on CB Data}
\author{Arthur Lui}
\date{\today} % \date{} to set date to empty

% MAIN %
\begin{document}

\maketitle

Two-dimensional t-SNE's were learned separately on each sample of the CB data.
Due to the size of the data (38636, 9555, and 4827 cells respectively 
in samples 1, 2, and 3), the Barnes--Hut algorithm was used to approximate
the exact t-SNE. Moreover, since our analyses revealed that 21 subpopulations
were required to adequately explain the data, and plotting 21 clusters 
was visually overwhelming, only clusters comprising more than 
5\% of cells in each sample are included in the graphs, though
the t-SNE's were fit on every cell of each sample.


\begin{figure}[t]
  \begin{center}
    \includegraphics[scale=1]{img/tsne_sample_1.pdf}
  \end{center}
  \caption{TSNE on sample 1 of CB marker expression data. 
  Each point represents a cell. Only cell subpopulations comprising more
  than 5\% of cells in the sample are included. Each subpopulation
  corresponds to a cluster obtained from the FAM and is marked by a
  unique marker and color.}
\end{figure}


\begin{figure}[t]
  \begin{center}
    \includegraphics[scale=1]{img/tsne_sample_2.pdf} \\
  \end{center}
  \caption{TSNE on sample 2 of CB marker expression data. 
  Each point represents a cell. Only cell subpopulations comprising more
  than 5\% of cells in the sample are included. Each subpopulation
  corresponds to a cluster obtained from the FAM and is marked by a
  unique marker and color.}
\end{figure}


\begin{figure}[t]
  \begin{center}
    \includegraphics[scale=1]{img/tsne_sample_3.pdf} \\
  \end{center}
  \caption{TSNE on sample 3 of CB marker expression data. 
  Each point represents a cell. Only cell subpopulations comprising more
  than 5\% of cells in the sample are included. Each subpopulation
  corresponds to a cluster obtained from the FAM and is marked by a
  unique marker and color.}
\end{figure}


% Uncomment if using bibliography:
% \bibliography{bib}
\end{document}
